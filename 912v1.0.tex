\documentclass[UTF8,12pt]{ctexart}
\usepackage{ctex}
\usepackage{amsmath}
\usepackage{graphicx}
\CTEXsetup[format={\Large\bfseries}]{section}
\title{\kaishu 912回忆版}
\author{by Shine Wong}
\date{12/22}

\begin{document}
\maketitle

	\section{数据结构}

	\subsection{判断题}

		\begin{itemize}

			\item[1)]$log^nn = \Omega(n^{logn})$。
			\item[2)]对一棵AVL树进行插入,则至多会引起$\Omega(logn)$次局部调整操作。
			\item[3)]对一个理想随机输入的序列进行快速排序,则在平均情况下以及最坏情况下都可以达到$O(logn)$的时间复杂度性能。
			\item[4)]在理想随机输入的情况下,尽管完全二叉堆的删除操作的最坏时间复杂度有$O(logn)$,平均时间复杂度仅为$O(1)$而已。
			\item[5)]跳转表每一个节点所对应的塔的平均高度为$O(logn)$。\\
			\item[6)]采用基于比较的算法,可以在$O(n)$的时间内找出序列的前10\%大的元素。
			\item[7)]对一有向图进行DFS,共有$k$条边被标记为 BACKWARD,则该图中未必有$k$个环路。\\
			\item[8)]败者树相对于胜者树,具有更优的渐进时间复杂度性能。\\
			\item[9)]相对于闭散列,开散列可以更好地利用数据的局部性。\\
			\item[10)]...remain to be added


		\end{itemize}

	\subsection{单向选择题}

		\begin{itemize}

			\item[1)]对一有向无环图,该图的拓扑排序序列恰好是DFS的$\underline{\hbox to 10mm{}}$\\
			A.\ 被发现的顺序\\
			B.\ 被发现的逆序\\
			C.\ 回溯的顺序\\
			D.\ 回溯的逆序

			\item[2)]如果基数排序底层采用不稳定的排序算法,则所得的结果$\underline{\hbox to 10mm{}}$,并且基数排序的稳定性$\underline{\hbox to 10mm{}}$\\
			A.\ 不再正确 \ 不再保持\\
			B.\ 不再正确 \ 仍然保持\\
			C.\ 仍然正确 \ 不再保持\\
			D.\ 仍然正确 \ 仍然保持

			\item[3)]逆波兰表达式$Blalala$的结果为2019,则中间缺失的操作符为\\
			A \ + \\
			B \ - \\
			C \ * \\
			D \ / \\
			E \ \^ \\
			F \ !

			\item[4)]对于一个权重分别是1,1,2,3,5,8,13,21的字符集构造Huffman编码树,其中最大的深度为\\
			A.\ 6\\
			B.\ 7\\
			C.\ 8\\
			D.\ 9

			\item[5)]含有$\underline{\hbox to 10mm{}}$个节点的真二叉树的数量,与2019对括号构成的合法表达式数量相同。\\
			A.\ 1009\\
			B.\ 1010\\
			C.\ 2019\\
			D.\ 4039

			\item[6)]对一模式串HHBFHHBFHHBFSHH,考虑改进的next数组,则$next[14] - next[0] = \underline{\hbox to 10mm{}}$\\
			A.\ 2\\
			B.\ 3\\
			C.\ 4\\
			D.\ 5

		\end{itemize}

	\subsection{证明题}

		已知一棵二叉搜索树的先序和后序遍历序列,是否可以构造出它的层次遍历序列?是则给出证明,否则给出一个反例。(5分)

	\subsection{程序设计题}

		给出二叉树节点BinNode的定义如下:\\

		\noindent class BinNode{\\
		public:\\
			\indent BinNode* parent;\\
			\indent BinNode* lc;\\
			\indent BinNode* rc;\\
			\indent int lsize;\\

			\indent BinNode* zig(BinNode* x);//绕当前节点顺时针旋转,仍然返回旋转后根节点的左子树\\
			\indent BinNode* zag(BinNode* x);//绕当前节点顺时针旋转,仍然返回旋转后根节点的右子树\\
		}

		\begin{itemize}

			\item[]

		\end{itemize}

\end{document}
